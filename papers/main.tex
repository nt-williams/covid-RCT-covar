\documentclass{article}

\usepackage{booktabs}
\usepackage[scaled]{beramono}
\renewcommand*\familydefault{\ttdefault} %% Only if the base font of the document is to be typewriter style
\usepackage[T1]{fontenc}
\usepackage[top=0.5in, bottom=0.5in, left=0.5in, right=0.9in]{geometry}
\usepackage{graphicx}
\usepackage{caption}
%\usepackage{biblatex}
\usepackage[utf8]{inputenc}
\usepackage[document]{ragged2e}

% bibliography format
\usepackage[authoryear]{natbib}
\bibpunct{(}{)}{;}{a}{}{,}

\bibliographystyle{apa}

%\addbibresource{references.bib}

\captionsetup[table]{skip=10pt}
\setlength{\parskip}{1em}
\setlength{\parindent}{0pt}

\title{
 {Efficiency gains in randomized trials for COVID-19 using specific covariates and performing variable selection with LASSO and random forests}\\
}
\author{Nicholas Williams, Iván Díaz, Michael Rosenblum}
\date{\today}

\begin{document}

\maketitle

\begin{abstract}
\end{abstract}

\section{Introduction}

\subsection{Covariate adjustment in RCTs}

\citep{covid19power}

\section{Statistical Estimands and Estimators}

\citep{vanderLaanRubin2006}
\citep{diaz2019improved}
\citep{diazOrdinal}

\begin{itemize}
  \item double-robust estimators
  \item consistent under correct specification of either treatment or outcome, guaranteed with the RCT.
  \item can use the LASSO and still get valid standard errors
\end{itemize}

\subsection{Survival outcome}

\newtheorem{surv}{Estimand}

For a time-to-event outcome, we consider two measures of treatment effect: the difference in the restricted mean survival time (RMST) and the difference in survival probability (also referred to as the risk difference, RD). Let $A$ be a binary treatment indicator where $1$ indicates assignment to the treatment arm and $0$ and indicates assignment to the control arm. In addition, let $T$ denote a patients survival time.

\begin{surv}[Difference in restricted mean survival time]
\label{rmst}
\end{surv}

The difference in restricted mean survival time is defined as the difference in the expected survival time between treatment and control arms when time is truncated at an arbitrary point, $\tau$.

\[E(\textnormal{min}\{T, \tau\} | A = 1) - E(\textnormal{min}\{T, \tau\} | A = 0)\]

\begin{surv}[Difference in survival probability]
\label{rmst}
\end{surv}
\[P(T > t | A = 1) - P(T > t | A = 0)\]

\subsection{Ordinal outcome}

\newtheorem{ord}{Estimand}

We consider two treatment effect estimands for an ordinal outcome: the average log-odds ratio (LOR) and the Mann-Whitney (MW) estimand. We define an ordinal outcome taking levels $1, ..., K$ and without loss of generality, we will assume that lower values of the outcome are preferred. 

\begin{ord}[Average log-odds ratio]
\label{lor}
\end{ord}

If the proportional odds assumption is correct, this value is equal to the corresponding coefficient from a proportional odds model (commonly referred to as ordinal logistic regression).

\[\frac{1}{K - 1} \sum_{j = 1}^{K - 1}\textnormal{log} \left( \frac{\textnormal{odds}(Y \leq j | A = 1)}{\textnormal{odds}(Y \leq j | A = 0)} \right)\]

\begin{ord}[Mann-Whitney]
\label{MW}
\end{ord}

The Mann-Whitney statistic is interpreted as the probability that a patient drawn at random from the treatment group has a greater outcome value than a patient drawn at random from the control group, with ties broken at random.

\[ P(Y^* > Y|A^* = 1, A = 0) + \frac{1}{2}P(Y^* = Y |A^* = 1, A = 0) \]

\section{Simulation Methods}

Our data generating distribution is based on a database of over 1,500 patients hospitalized at Weill Cornell Medicine New York Presbyterian Hospital prior to 15 May 2020. For a description of the clinical characteristics and data collection methods of the initial cohort sampling, see \citet{goyalNEJM}.

For both outcome types, we evaluated the following auxillary covariates: age, sex, BMI, smoking status, whether the patient required supplemental oxygen within three-hours of presenting to the Emergency Department, number of comorbidities (diabetes, hypertension, COPD, CKD, ESRD, asthma, interstitial lung disease, obstructive sleep apnea, any rheumatological disease, any pulmonary disease, hepatitis or HIV, renal disease, stroke, cirrhosis, coronary artery disease, active cancer), number of relevant symptoms, presence of bilateral infiltrates on chest x-ray, dyspnea, and hypertension. Patient data were resampled with replacement to generate 1000 datasets of size $n = 100, 250, 500 \textnormal{ and } 1500$. For each dataset, a treatment value was draw for each patient with probability 0.5 from a Bernoulli distribution.

For each generated dataset, we evaluated the estimands of interest with the following conditions: unadjusted;  adjusting for each individual covariate; adjusting for all covariates; adjusting for all covariates using the LASSO \citep{tibLASSO}; adjusting for all covariates using random forests. In scenarios using the LASSO or random forest, all nuisance parameters were estimated with LASSO while only the outcome and censoring models were estimated using random forests and an intercept only model was used to estimate the propensity. In addition, the use of random forests was evaluated with and without cross-fitting \citep{cfVictor}. All simulation scenarios were also repeated after shuffling covariates to remove any association between the covariate and observed outcomes.

\subsection{Survival outcome}

For the survival outcome, the outcome is defined as the time from hospitalization to intubation or death. We focused on estimating the difference in RMST at 14 days after hospitalization and the difference in survival probability at 7 days after hospitalization. A positive treatment effect was simulated by adding an independent random draw from a $\chi^2$ distribution with varying degrees of freedom (to produce different effect sizes) to each patient's observed survival time in the treatment arm. These effect sizes translate to a difference in RMSTs of 0.55 and 1.07, and RDs of 0.04 and 0.10, respectively. To simulate uninformative censoring, 5\% of the patients were selected at random to be censored with censoring times drawn from a Uniform distribution, $U(1, 14)$. 

\subsection{Ordinal outcome}

We defined a six-level ordinal outcome at 14 days post-hospitalization based on the WHO Ordinal Scale for Clinical Improvement (INSERT CITATION). The categories are as follows: 0, discharged from hospital; 1, hospitalized with no oxygen theraphy; 2, hospitalized with oxygen by mask or nasal prong; 3, hospitalized with non-invasive ventilation or high-flow oxygen; 4, hospitalized with intubation and mechanical ventilation; 5, dead.

A positive treatment effect was simulated by subtracting from the treatment arm an independent random draw from a four-parameter Beta distribution with support $x \in [0, 5]$, $\textnormal{Beta}(\alpha, 15)$; we sampled from distributions with the $\alpha$ parameter set to 1.5 and three to generate different effect sizes. These effect sizes corresponds to an LOR of 0.22 and 0.60, an a Mann-Whitney value of 0.49 and 0.46 respectively. 

\subsection{Performance criteria}

\section{Simulation Results}

Tables containing the comprehensive results of simulations are presented in the Appendix.

\begin{itemize}
  \item Supplemental O2 provides the greatest efficiency gains
  \item As sample size increases, the LASSO and RF provides largest efficiency gains
  \item Without variable selection, adjusting for non-prognostic variables results in efficiency losses
  \item The LASSO maintains efficieny when covariates are not predictive of the outcome
  \item The LASSO and RF handle zeros in an ordinal outcome while GLM becomes unstable
  \item Without cross-fitting, RF overfits but remains un-biased because the propensity is still correctly specified 
  \item With cross-fitting, RF...
\end{itemize}

\section{Recommendations}

\section{References}

%\printbibliography
\bibliography{references}

\appendix

\end{document}